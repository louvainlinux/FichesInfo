\documentclass{../fiche}

\begin{document}

% Titre
\titlellnux{6}{\LaTeX}
\vspace{-.4cm}

% Contenu
\paragraph{Knuth: The Art of Computer Programming}
En 1973, \textsc{Donald Knuth} publia son \oe uvre de référence "\textit{The art of computer programming}" dont aucun bon informaticien ignore l'existence. Quand il commença à la rédiger, il se rendit compte qu'aucun logiciel de typographie ne produisait des textes suffisamment clairs et structurés. Il décida donc de créer sa propre solution qui est désormais connue sous le nom de \TeX{} (voir même plain\TeX). Le génie de Knuth ne vient pas sans un certain perfectionnisme. Pour en donner un exemple, il reformulait ses phrases jusqu'à ce qu'elles harmonisaient avec ses alinéas tellement il visait l'unité de la forme et du contenu.
Le premier qui trouve une faute dans ces livres volumineux (même une virgule mal placée) sera payé un cent par M. Knuth. Le deuxième le double et ainsi de suite.

À moins que vous soyez aussi perfectionniste que Knuth, nous ne vous conseillons pas d'écrire vos documents en \TeX. Cela revient à de le programmation pure et dure. Il y a aujourd'hui des logiciels plus simples pour faire la même chose.

\paragraph{LaTeX}
\LaTeX{} se base sur \TeX{} tout en étant beaucoup plus facile à gérer. Vous dites au programme que ceci est un chapitre et cela une section ou sous-section et \LaTeX{} va s'occuper de les mettre dans la bonne grandeur, de les numéroter et de les mettre dans la table des matières. \LaTeX{} permet de réaliser des documents de très grandes qualités, allant jusqu'à effectuer de nombreux calculs pour adapter au mieux le nombre de mots par ligne, réaliser des césures où et quand il le faut, gérer les fins de page pour ne pas y insérer un titre seul ou ajouter un nouvelle page uniquement pour 3 mots, placer des images et tableaux là où c'est censé être le plus agréable, etc.

Ce que \LaTeX{} a un commun avec \TeX{} et ce qui le différencie des autres programmes comme LibreOffice, Word etc., c'est que vous ne travaillez pas sur le document final. Vous travaillez sur du code (une syntaxe très simple, n'ayez pas peur) qu'il faut "compiler" pour en faire typiquement un fichier PDF. On dit que \LaTeX{} n'est pas \textsc{Wysiwig} (\textit{What you see is what you get}). % Une fois qu'on s'est habitué, il y a un énorme avantage à cela.
%En faisant référence à l'acronyme \textsc{Wysiwig}, ses défendeurs disent que
\LaTeX{} est d'ailleurs décrit comme un \textsc{Wygiwym} (\textit{What you get is what you meant}).

Vous avez tous sans doute connu des situations ou votre fichier ODT ou DOC ne veut pas obéir à vos ordres. Vous savez ce que vous voulez obtenir, mais le texte se déplace soit trop à gauche, soit trop à droite. Les alinéas ne fonctionnent pas bien, les tableaux sont coupés et distribués sur deux pages, les équations compliquées sont difficiles à composer, le nombre de clics de souris à faire vous arrêtent dans votre lancée... Si vous mettez en commun des travaux de groupes réalisés sur des ordinateurs différents, ces problèmes se multiplient.

En travaillant avec \LaTeX{}, vous n'aurez plus ce genre de problèmes. Vous réfléchirez surtout à la structure de votre texte et beaucoup moins à sa mise en page. Vous pouvez changer les marges, la police et la grandeur du texte tout à la fin de votre travail. \LaTeX{} se base sur la structure de l'information et va trouver soi-même comment arranger l'information sur les pages tout en ajustant les références au nouveau document.

Un autre avantage est qu'avec \LaTeX{}, il est particulièrement facile de rédiger des formules mathématiques complexes sans se tromper. Presque tous vos syllabi de sciences exactes ainsi qu'une bonne partie de vos livres de référence sont rédigés en \LaTeX{}. Sans compter le nombre de mémoires et autres travaux imposants où \LaTeX{} devient presque indispensable pour gérer facilement une bibliographie, de nombreuses parties dans le programme, la gestion des références, des tableaux, formules, images, etc. tout en ayant, au final, un rendu d'une qualité remarquable.

\paragraph{Bibtex}
Les Maths c'est bien, mais quid des sciences humaines? Là aussi, \LaTeX{} est très intéressant. Tout d'abord, précisions qu'il y a bien-sûr moyen d'écrire des documents \LaTeX{} avec des polices imposées comme \textit{Times New Roman} et compagnies.
Les avantages en ce qui concerne la structuration du document restent présents qu'il s'agisse d'un texte mathématique ou non. En plus de cela, il y a un autre avantage majeur : BiB\TeX{} est là pour s'occuper de vos sources.

Si vous trouvez des articles scientifiques dans les bases de données en ligne, les PDF sont accompagnés de fichiers \textit{BIB} qui contiennent de manière standardisée les sources du document tel que les auteurs, le journal, le numéro... BiB\TeX{} s'arrange pour mettre dans la bibliographie tous les articles auxquels il y a une référence dans le texte et uniquement ceux là. Les conventions d'écriture des sources sont automatiquement respectées: les \textit{op. cit.}, \textit{ibidem}, etc. sont automatiquement gérés. Une chose en moins à garder en tête.
À noter également que grâce au \textit{paquet} Babel, il est possible de spécifier en quelle langue le document est écrit pour automatiquement adapter le style du document en fonction des normes de la langue.

\paragraph{Les paquets et les classes}
Il y a des milliers d'extensions (paquets) pour plus ou moins tous les besoins. Il y a également des classes qui permettent de réaliser différents types de documents (article, livre, lettre, CV, etc.) :
\begin{itemize}
	\item La classe \texttt{moderncv} produit des CVs efficaces et bien structurés ainsi que des lettres de motivation.\\ Il suffit de remplir les bons champs afin de produire une CV d'un style idéal.
	\item La classe \texttt{beamer} est là pour produire des slides.
	\item Pour les profs, le paquet \texttt{alterqcm} génère facilement des \textsc{Qcm}.
	\item Ti\textit{k}Z est là pour dessiner des petits schémas vectoriels directement en \LaTeX{}.
	\item Mais également beaucoup d'autres pour gérer des équations, des tableaux spéciaux, inclure des PDF, des formules chimiques et des représentations de molécules, réaliser des schémas électriques, des arbres en informatiques, etc.
\end{itemize}

\end{document}
	
