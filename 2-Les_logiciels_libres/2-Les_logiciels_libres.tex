\documentclass[10pt]{../fiche}

\begin{document}

% Titre
\titlellnux{2}{Les logiciels libres}

\vspace*{0.5cm}
Voici déjà notre 2\up{ième} fiche informative. Dans celle-ci vous découvrirez ce que sont les logiciels libres, leurs avantages ainsi qu'une correspondance entre logiciels propriétaires (dit privateurs) et logiciels libres afin de vous aider à changer.

\section*{Qu'est-ce qu'un logiciel libre?}

Pour rappel, un logiciel libre est caractérisé par plusieurs points :
\begin{itemize}
\item Il est libre d'utilisation: toute personne peut l'utiliser à toutes les fins possibles
\item Il est libre d'étude: son code source (recette de fabrication) est accessible par tout ceux qui veulent
\item Il est libre de modification: toute personne désirant ajouter un ingrédient dans la recette peut le faire
\item Il est libre de redistribution: toute personne peut recréer le même programme et le distribuer
\end{itemize}

Pour éviter qu'un logiciel libre passe du coté obscur de la force, il est également soumis à une licence d'utilisation. Il en existe plusieurs dont la licence CC (\textit{Creative Commons}) ou la licence GPL (\textit{General Public Licence}), pour la plupart regroupées sous le principe \textit{Copyleft}. Chacune de ces licences autorisent un certain type d'utilisation. Pour toutes œuvres regroupées sous \textit{Copyleft}, tout dérivé est obligé de rester sous la même licence, cela permet ainsi de préserver le coté libre d'une œuvre.

\section*{Quelques avantages d'un logiciel libre}

Les logiciels libres ont de nombreux avantages.

Comme un logiciel libre est le résultat du travail de nombreuse personnes, il en est d'autant plus efficace. Puisque chacun à la possibilité de corriger les bugs, ceux-ci sont d'autant plus vite éliminés. S'il manque, une fonctionnalité, celle-ci sera rapidement rajoutée. Et comme le nombre de développeurs est souvent élevé et qu'ils ont tous des besoins spécifiques, cela assure au programme un très grand nombre de fonctionnalités avancées et/ou un grand nombre de programmes différent.

De plus, on est assuré d'une sécurité au point de vue des données. Toutes failles de sécurité sont également corrigées très rapidement.

Ensuite, les logiciels libres sont généralement unis pour respecter des standards. Ainsi, les logiciels vont essayer de format un ensemble cohérent, partager des ressources et utiliser des formats de fichiers lisibles par d'autre logiciels (cela permet de toujours conserver la lisibilité des données même si le logiciel d'origine est obsolète). %% "essaye" car KDE est un bon exemple de l'utilisation des standards à leur manière ;)

Quatrièmement, la plupart des logiciels libres sont gratuits.

\section*{Quelques logiciels propriétaires et leurs équivalents libres}

Je vais maintenant vous donner une petite liste de logiciels propriétaires avec leur équivalent libres multi-plateforme (pour \textit{Windows}, \textit{MacOSX} et \textit{GNU/Linux}). %% ou faire un tableau?
\begin{itemize}
	\item La suite \textit{M\$ office} remplaçable par la suite \textit{LibreOffice/OpenOffice.org}
	\item \textit{Outlook Express} remplaçable par \textit{Mozilla Thunderbird}.
	\item On peut utiliser \textit{Gimp} à la place du célèbre logiciel \textit{Photoshop}.
	\item Comme navigateur web, à la place de \textit{Safari} ou \textit{Internet Explorer}, vous pouvez utiliser \textit{Mozilla Firefox}. \textit{Google Chrome}, quant à lui, se base sur \textit{Chromium} en ajoutant de petits bouts de code propriétaire (donc on ne sait pas exactement ce qu'il fait).
	\item Comme lecteur vidéo, à la place de \textit{Windows Media Player} et \textit{RealPlayer}, il y a bien-entendu \textit{VLC}.
	\item À la place d'\textit{Adobe Reader}, vous pouvez utiliser \textit{Sumatra} (\textit{Windows}) ou \textit{Evince} (\textit{Linux}) pour lire vos PDF et bien d'autres formats. %% Xpdf, c'est un peu vieux...
	\item Pour ceux qui font de la modélisation 3D sur \textit{Maya}, \textit{Blender} est là pour vous.
	\item Avis aux fans de musiques, au lieu de passer par \textit{iTunes}, vous pouvez utiliser \textit{Clementine}. % aux possesseurs d'iPod
	\item Pour les utilisateurs de \textit{Winrar}, il existe \textit{7-zip} (\textit{Windows}), ou \textit{p7zip} (\textit{Linux}).
	\item \textit{Matlab} à également un équivalent, \textit{Octave}.
	\item Pour les créateurs de musiques, il y a aussi de quoi se régaler avec \textit{Ardour}, \textit{LLMS}, \textit{Audacity}, \textit{RoseGarden}, \textit{TuxGuitar}, \textit{Hydrogen} et bien d'autres.
\end{itemize}
Pour plus d'informations sur ces logiciels ou comment les installer, n'hésitez pas à taper leur nom dans un moteur de recherche ou à passer nous voir un lundi en permanence ;)\\
Sur \url{framasoft.net/rubrique2.html}, vous aurez une liste de plus de 1600 logiciels libres classés par domaine d'application.




\end{document}
