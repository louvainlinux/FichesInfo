\documentclass[12pt]{../fiche}

\begin{document}

\titlellnux{14}{C'est quoi Linux ?}

Coucou petit ami du Libre !!
\vspace{0.8em}

Maintenant que tu connais par coeur notre fiche sur le Libre, on s’est dit que ça serait cool que tu découvres le logiciel libre le plus connu et le plus utilisé au monde, à savoir Linux, fin plus précisément GNU/Linux pour les puristes.
\vspace{0.8em}

Alors c’est quoi GNU/Linux ? Vous voyez Windows et Mac OS, c’est pareil mais en beaucoup mieux. C’est donc un système d'exploitation dont la première version remonte à l’antiquité, autrement dit en 1991. Ca ne nous rajeunit pas tout cela. D’un côté, il y a Linux qui est le kernel ou noyau, en gros le logiciel qui permet de faire la liaison entre le matériel physique du PC et les autres logiciels. De l’autre côté, il y a GNU qui est un projet datant lui de 1984 (on se sent encore plus vieux) qui, ajouté à Linux permet d’avoir un système d'exploitation complet. Mais pour des questions de facilité, on dit Linux pour désigner le tout (les informaticiens sont les plus fainéants d’entre nous).
\vspace{0.8em}

“Mais vous parlez toujours d’Ubuntu quand vous passez en TDK !” En somme, c’est une distribution Linux, autrement dit c’est Linux auquel les développeurs ont choisi une interface graphique particulière et un ensemble de logiciels par défaut comme Firefox ou Chromium (version Libre de Google Chrome), … Mais aussi chaque distribution a sa propre philosophie. Ubuntu, elle, est la distribution la plus utilisée dans le monde pour les ordinateurs portables et de bureau, du coup on parle beaucoup d’elle vu que c’est celle qu’on installe le plus souvent et c’est aussi une des plus simple à utiliser, plus simple que Windows même, selon certains.
\vspace{0.8em}

Ok, c'est bien beau tout ça, mais à quoi est-ce que ça me sert concrètement?
Cette liste non exhaustive contient quelques unes des raisons les plus souvent rencontrées:
\begin{itemize}
\item Déjà c’est Libre, ca devrait suffire à te convaincre;
\item C'est gratuit, enfin légalement gratuit. Pas comme les logiciels crackés que bien sûr tu n’as jamais utilisé :p ;
\item Une fois les programmes installés, ils se mettent automatiquement à jour;
\item Comme il y a des centaines de distributions, tout le monde y trouve son bonheur;
\item En parlant de bonheur, tu n’es pas limité ou imposé par un éditeur sur l’interface ou les logiciels installés sur ton PC, tu choisis ce qui te plait le plus;
\item Linux est très fiable au niveau sécurité. Donc, pas de virus qui s’installe tout seul, ainsi tu as même pas besoin d'antivirus qui te rappelle à haute voix que “La base virale VPS a été mise à jour”;
\item Linux fonctionne très bien sur des vieux PC, petits netbooks et autres
\item Linux est le système d’exploitation le plus utilisé au monde dans les supercomputers, serveurs, smartphones sous Android, ... 
\end{itemize}
\vspace{0.8em}

Si tu as envie d’en apprendre plus ou de l’installer sur ton PC à côté de ton Windows ou même à la place de Windows, n’hésite pas à passer à notre kot lors de nos permanences informatiques, tous les lundi de 18h30 à 22h. Nous serons ravis de t’accueillir pour t’aider et répondre à tes questions.

\end{document}
