\documentclass[12pt]{../fiche}

\begin{document}

% Titre
\titlellnux{7}{Open-source hardware}

% Contenu
Le monde du libre ne se limite aucunement aux logiciels ou même à l'informatique en général.\\
On peut en effet appliquer tout ses concepts à tout matériel physique.

\paragraph{\textit{C'est quoi?}}
La matériel, peu importe lequel, contient toujours quelque chose de plus que des matériaux. Il contient des idées que l'on peut résumer au plan. Une voiture par exemple, c'est plus qu'un amas de métaux, plastiques et autres. Si l'on peut posséder une voiture, on ne possède généralement pas son plan.

Le matériel libre, c'est lorsque le plan est mis à entière disposition du public. On peut l'étudier, l'utiliser, le modifier ou encore le dupliquer. Bref, on peut en faire ce qu'on veut si ce n'est se l'approprier.

\paragraph{\textit{C'est bien?}}
Le matériel libre permet une large contribution du public, une diminution drastique du coup de fabrication ainsi que pleins d'autres avantages. Lors de la conception, il n'y a plus qu'un seul objectif, fournir le meilleur produit possible sans aucune contrainte de rentabilité.

Un autre point important est qu'il est beaucoup plus facile d'obtenir ou de réaliser une version personnalisée vu que l'on a la possibilité d'étudier le système que l'on veut modifier.

De plus, cela évite certaines dérives. Dans le monde des PC, par exemple, certains constructeurs n'hésitent pas à souder des composants dans le seul but de flouer le consommateur. Ainsi, en cas de pépin, pas le choix, obligé de se tourner vers le constructeur et de payer un maximum.

\paragraph{\textit{Comment?}}
Ok, tout ça c'est bien mais moi, comment je fais pour en profiter ? Tout ce qu'il vous manque, c'est un constructeur capable de produire ce que vous voulez. En fait, n'importe qui ayant les moyens techniques peut le faire, forcément, ça augmente les possibilités et réduit les coûts.

\paragraph{\textit{Le Raspberry Pi}}
Le Raspberry Pi est un ordinateur de la taille d'une carte de crédit dont les plans sont sous licence libre. N'importe qui possédant les moyens techniques de le produire peut le faire. C'est grâce à cela qu'il est possible de le trouver sur le commerce pour à peine une vingtaine d'euros.

\paragraph{\textit{DIY drones}}
Lisez ``Do it yourself drones'' ou ``faites votre drone vous-même''. diydrones.com est un site internet où l'on peut partager les plans pour la conception de ``robot volant'' ainsi que les pilotes (la partie logicielle) ou encore chercher de l'aide pour le conception de son propre robot.

Ce site permet de se procurer des engins dont les plus onéreux sont en dessous de la barre des mille euros. Un exploit quand on sait que le prix est très souvent supérieur à dix mille euros dans le commerce privé.

\paragraph{\textit{L'imprimante tridimensionnelle}}
Une imprimante tridimensionnelle permet de créer des objets en 3D. Cette technologie s'est très fortement développée ces dernières années, de plus en plus précise et utilisant des matériaux de plus en plus variés.

Ce type d'outil permettrait des créations sans limite. Vous créez ou téléchargez le plan, imprimez et c'est fait ! Cela donnerait un nouveau sens au matériel libre dans la mesure ou la création des objets ne poserait plus de problème technique.

Une large communauté travaille actuellement sur la conception d'une imprimante tridimensionnelle libre. Elle pourrait, à quelques pièces près, s'imprimer elle-même! Une fois qu'elle sera conçue, il suffira de demander à un ami qui la possède déjà de la réimprimer à partir des matières premières pour l'obtenir.

\end{document}
