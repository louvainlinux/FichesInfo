\documentclass[12pt]{../fiche}

\begin{document}

\titlellnux{16}{Wikipédia}
Salut à toi futur ou actuel ami du Libre ! 
\vspace{0.8em}

Tu connais bien sûr wikipédia, LA référence incontestée pour toutes les recherches d'informations.
Source (parfois) principale de vos travaux et des cours de vos profs (hum...),
cette encyclopédie révolutionnaire fut au coeur de nombreuses controverses.
Malgré cela, elle a grandi année après année pour atteindre sa suprématie d'aujourd'hui.

Tu l'utilises tous les jours, tu as surement entendu parler de son mode
d'édition collaboratif (n'importe qui peut participer à la rédaction d'un
article sur wikipédia) mais connais-tu vraiment cette encyclopédie ?

\vspace{-0.3cm}
\section*{Wikipédia en quelques chiffres}

Sur Wikipédia français, il y a...
\begin{itemize}
    \item près de 2 millions d'articles
    \item près de 3 millions de contributeurs
    \item plus de 24 millions de visiteurs uniques par mois (5\ieme site le plus visité de France)
\end{itemize}
\medskip

Au total (toutes les langues confondues), il y a...
\begin{itemize}
    \item plus de 46 millions d'articles
    \item près de 32 millions de contributeurs
    \item environ 100 millions d'heures de travail pour rédiger le contenu
\end{itemize}
\medskip

Par rapport à wikipedia, toutes les autres encyclopédies font donc figure de poids plumes.

\vspace{-0.3cm}
\section*{Critiques}

Le statut de Wikipédia en tant que source de référence est un sujet de
controverses, en particulier à cause de son système de rédaction ouvert à tous.
L'audience grandissante de Wikipédia a conduit un grand nombre de personnes à
formuler des avis critiques sur la fiabilité des informations présentées dans
cette encyclopédie.

Les principales critiques portent sur :
\begin{itemize}
    \item l'anonymat des contributeurs ;
    \item l'absence de filtrage des éditeurs et de comité de validation ;
    \item les problèmes posés par la neutralité de point de vue ;
    \item la vulnérabilité face aux sabotages, « vandalismes » dans le jargon de Wikipédia ;
    \item la communauté des contributeurs.
\end{itemize}

D'autres critiques se révèlent plutôt positives. Ainsi, en juin 2009, le
philosophe français Bernard Stiegler estime que Wikipédia, « passage obligé
pour tout utilisateur d'Internet », est un « exemple frappant d'économie de la
contribution » et que l'encyclopédie « a conçu un système d'intelligence
collective en réseau ».

Des études ont été menées sur la qualité du contenu proposée par Wikipédia, et
des comparaisons effectuées avec d'autres encyclopédies. Ces évaluations
fournissent généralement des conclusions positives pour Wikipédia, mais ces
résultats font aussi l'objet de critiques.

\vspace{-0.3cm}
\section*{Licence}
Tout le contenu de wikipédia est disponible sous la licence libre Creative
Commons - Attribution - Partage dans les mêmes conditions.
Concrètement, cela signifie que vous pouvez copier du contenu de wikipédia, à
condition de citer la source et de remettre le contenu sous la même licence.

\vspace{-0.3cm}
\section*{Les projets wikimedia}
Autour de wikipedia, il y a toute une série de projets qui fonctionnent sur le même principe et qui sont aussi gérés par la fondation wikimédia:
\begin{itemize}
    \item Wiktionnaire: un dictionnaire,
    \item Wikiquote: un répertoire de citations,
    \item Wikibooks: un fond de livres dans le domaine public, mis en ligne,
    \item Wikisource: un fond de textes libres de droits,
    \item Wikispecies: un répertoire centrale d'espèces pour la taxonomie,
    \item Wikimedia Commons: une médiathèque en ligne,
    \item Wikinews: un site d'actualités
    \item Wikiversité: un recueil de contenus pédagogiques,
    \item Wikidata: une base de connaissances (sous forme utilisable par des machines),
    \item et bien d'autres...
\end{itemize}

\vspace{-0.3cm}
\section*{Et toi ?}

Tu es expoert dans un domaine et tu as envie de partager tes connaissances au plus grand nombre ?
Tu a envie d'apporter ta pierre à ce monument d'internet qu'est wikipédia ?
Tu rêves d'avoir le titre de "rédacteur d'encyclopédie" sur ton CV ?
Tu trouves que la page sur le kaléidoscope/les paraphilies/La Baraque/les Kot-à-projet est à améliorer ?

Ça tombe bien ! En S11, l'UCL organise une semaine de contribution à Wikipédia.
Que tu sois novice complet ou contributeur confirmé, analphabète en informatique ou alter ego de Linus Torvalds,
ce tu es le ou la bienvenu·e.

Concrètement, toutes les infos sur sur la page du projet\footnote{
    \url{https://uclouvain.be/fr/mondesnumeriques/projet-wikipedia.html}
},
avec notamment:
une liste de thèmes proposés, avec pour chacun une séance collaborative
d'édition; et des séances "libres" (les horaires et lieux seront sur notre site web)
où chacun peut écrire sur ce qu'il souhaite,
et poser des questions à des contributeurs plus aguerris.

Il n'est pas nécessaire d'écrire du texte pour contribuer: ajouter des liens
dans les pages, réparer les liens cassés, ajouter des références ou des
illustrations... il y a plein de façons de contribuer !


\paragraph{Sources}
\begin{itemize}
    \item \url{https://fr.wikipedia.org/wiki/Wikipédia}
    \item \url{https://fr.wikipedia.org/wiki/Wikimedia_Foundation}
    \item \url{https://stats.wikimedia.org/}
\end{itemize}



\end{document}
