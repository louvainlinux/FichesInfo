\documentclass[10pt]{../fiche}

\begin{document}

% Titre
\titlellnux{4}{Faire de l'argent avec des logiciels libres}

% Contenu
\vspace{15pt}
Contrairement à ce que tu penses peut-être, il est tout à fait possible pour une personne ou une entreprise de distribuer uniquement des logiciels libres, et d'en vivre. Comment ? C'est ce que nous allons voir.
\section*{L'économie du libre}
Selon Wikipédia, l'économie rassemble les activités humaines tournées vers la production, l'échange, la distribution et la consommation de biens et de services.

Dans le cas des logiciels propriétaires, il est facile de comprendre que leur économie se base sur la production et la distribution de logiciels payants. %Leurs logiciels sont artificiellement raréfiés en mettant en place des systèmes de protection contre le copiage de plus en plus envahissants. 
Les clients doivent signer des licences onéreuses qui
leur donnent seulement le droit d'utiliser le logiciel dans des circonstances délimitées par le vendeur.
Il est, par exemple, souvent illégal de louer ou prêter un logiciel, pourtant légalement acquis,
pour que quelqu'un d'autre puisse le tester alors qu'il n'y a pas de problème pour prêter un livre (sauf s'il est sur votre \textit{iThing}). Les logiciels libres veulent éviter toutes ces restrictions et encouragent même le copiage!

Comment une entreprise peut-elle être rentable si ses logiciels sont libres et donc, par définition, librement distribuables et souvent gratuits ? La réponse se trouve dans la dernière partie de la définition de Wikipédia: l'économie du libre ne se base pas sur des \textit{biens}, mais plutôt sur des \textit{services}. Pour comprendre comment cela fonctionne, il faut se concentrer sur le monde
des logiciels professionnels utilisés en entreprise. Ici, les coûts de la licence
ne représentent pas la totalité des dépenses : on les achète presque toujours avec
des contrats de maintenance et des formations.
L'éditeur ne gagne donc pas d'argent sur la distribution du logiciel, mais il a d'autres manières de rentabiliser son activité. Nous allons en voir trois exemples.

\section*{Les contrats de service}
Une entreprise qui veut faire tourner toutes ses bases de données internes (gestion du personnel,
gestion des clients, gestion des stocks, comptabilité, etc.) sur un seul logiciel a intérêt à bien faire son choix puisqu'il lui sera difficile de changer par après. %, il s'agit d'une décision stratégique.
Dans tous les cas, elle ne va pas se contenter "d'acheter" ce logiciel et de se débrouiller toute seule.
Typiquement, elle signe un contrat de garantie/service/maintenance. % avec ceci? Ce genre de contrat fonctionne indépendamment du caractère libre ou propriétaire du logiciel en question.
Le contrat inclut au-moins l'installation sur place mais aussi, il spécifie une personne de contact et les détails de l'intervention garantie en cas d'urgence ou de mise à jour.

Tant que l'entreprise cliente peut se tourner vers un "vendeur" fiable qui fait preuve de références
positives dans le passé, elle ne s'occupe pas trop du caractère libre du logiciel.
Comme exemples concrets, il y a \textit{Odoo} et \textit{RedHat}.

\section*{La programmation sur commande}
La programmation sur la demande d'un client %particulier
s'est toujours avérée intéressante puisqu'on
peut lui facturer l'entièreté des coûts. Le code ainsi produit reste confidentiel dans le
modèle propriétaire. Dans bien de cas, même l'acheteur ignore le fonctionnement
interne du programme qu'il a commandé et le vendeur est interdit de l'utiliser dans d'autres contextes.
Ce type de licences emprisonne les résultats du travail et produit des redondances incroyables.

Par contre, quand le logiciel conçu est destiné à être libre, on facture souvent uniquement une partie
des coûts au client. Ce dernier sera incité à commander son code en open source puisqu'il paye moins. %% il n'y a pas que l'argent, il peut aussi profiter d'un produit fort utilisé (sous-entendu juste en dessous? p-ê le préciser?), pouvant être adaptable (au-dessus) et évoluant rapidement (plus bas)
En même temps, le vendeur peut se permettre de facturer moins cher parce que ce code open source
améliorera son portfolio de solutions disponibles. Il pourra vendre plus de contrats de maintenance.
Le logiciel libre peut donc évoluer beaucoup plus vite qu'un logiciel propriétaire.

\section*{Les partenariats: producteur - installateur - client} % implémenteur? vendeur?
Les partenariats perfectionnent la synergie entre le producteur et l'utilisateur du logiciel
libre. L'entreprise partenaire s'occupe de tous les services à fournir sur place comme l'installation, la formation et
la traduction de la documentation. Parfois elle étend le code source du logiciel pour ajouter des modules.
Elle profite de la crédibilité du producteur et reçoit une partie des revenus du contrat de service pour son travail.
Le producteur se charge de fournir une équipe de base qui développe le noyau du programme et
prend les décisions stratégiques à long terme. Il est heureux que les installateurs
distribuent son produit. Chacun s'occupe de ce qu'il sait % ou 'peut'?
faire le mieux, une situation \textit{win-win}.
Cette collaboration est beaucoup plus difficile
quand il s'agit d'un logiciel propriétaire, la confidentialité freine l'échange et le progrès.
\end{document}
