\documentclass[10pt]{../fiche}

\begin{document}

% Titre
\titlellnux{5}{Le libre, pas que des logiciels}

% Contenu
Le monde du libre ne se limite aucunement aux logiciels ou même à l'informatique en général.\\
Parcourons certains exemples plus ou moins connus pour illustrer ce que nous entendons par cela.

\paragraph{\textit{Wikipedia}}
Tout le monde l'utilise de temps en temps. Cette encyclopédie libre a été fondée par \textsc{Jimmy Wales} en 2001.
Chacun peut contribuer aux articles; d'où la taille phénoménale.
On y trouve plus d'un million d'articles en français et plus de quatre millions en anglais, soit plus que toutes les encyclopédies commerciales combinées.
La critique principale est que le contenu serait non fiable.
Est-de vraiment le cas? C'est vrai qu'on ne connaît pas toujours les auteurs des articles, mais bien leurs sources respectives.
C'est là dessus que se base le contrôle du contenu et franchement, ça marche.
S'il convient de critiquer \textit{Wikipedia}, c'est probablement parce que les articles ont tendance à devenir trop longs et que des sujets tels que l'informatique ou les séries TV
sont largement surreprésentés en raison des intérêts personnels des auteurs.

En plus des encyclopédies, il existe aussi \textit{Wikimedia Commons} une base de données multimédia libre
qui est la source des illustrations que vous trouvez dans les articles.
Parmi les autres projets liés, on trouve \textit{Wikiquote}, un recueil de citations et \textit{Wiktionnaire}, un dictionnaire universel.

\paragraph{\textit{Open Street Map}}
Le principe est le même que pour \textit{Wikipedia}, tout le monde peut aider à construire ces cartes.
Faites le test et comparez-les avec celles de \textit{Google Maps} ou encore des cartes imprimées.
Dans la plupart des cas, les cartes de \textit{Open Street Map} seront beaucoup plus détaillées.
Il reste encore à développer un algorithme d'optimisation de chemin.

\paragraph{La musique libre}
Il y en a de deux genres. Parlons d'abord des \textit{Creative Commons} (\textit{CC}). Ce sont des licences grâce auxquelles l'auteur de la pièce (musique ou autre) peut vous permettre
d'utiliser son \oe uvre tant que certaines conditions restent remplies. Par exemple, la mention \texttt{BY} veut dire qu'il faut mentionner l'auteur original, \texttt{NC} permet son l'utilisation
uniquement dans des circonstances non commerciales et \texttt{SA} vous permet de réutiliser ce matériau tant que votre nouveau produit reste sous la même licence libre.
On parle parfois d'une licence \textit{copyleft} puisqu'elle utilise les mêmes mécanismes juridiques que le \textit{copyright} pour atteindre le but contraire. Vous trouvez plein de musiques libres sur \url{www.jamendo.com}

En plus de cela, il y a la musique dont le copyright a expiré. Actuellement, c'est le cas \textit{70 ans après la mort de l'auteur}.
Il n'est pas très utile de retenir ce chiffre puisqu'il se fait scandaleusement prolonger à chaque fois que des \oe uvres comme \textit{Micky Mouse}
ou les chansons d'\textit{Elvis} risquent de tomber dans le domaine public.
Néanmoins, une chose est sure. La plupart du patrimoine culturel qu'est la musique classique est aujourd'hui libre de droits d'auteurs.
C'est entre autres pour cette raison qu'on en entend autant dans la publicité et les films.
C'est uniquement l'ensemble des notes de musique et non l'interprétation par un orchestre particulier (ni la partition papier dans certains cas) qui est libre.
Le site web \url{www.musopen.org} présente donc un catalogue de musique classique rejoué avec l'intention de rendre ces interprétations libres.
Nous vous conseillons de les télécharger dans le format \texttt{flac} (\textit{free lossless audio codec}) en raison de la qualité supérieure aux \texttt{mp3}.

\paragraph{Les livres}
Ici on trouve de nouveau les scans de livres historiques sous domaine public, mais aussi quelques romans publiés sous licence \textit{CC}.
Mentionnons à titre d'exemple les romans policiers et de science-fiction de \textsc{Cory Doctorow}.

\paragraph{Les cours}
Ceci est particulièrement intéressant pour tout étudiant: Presque toutes les universités de renommée mondiale mettent une grande partie de leurs cours et conférence en ligne.
Ces vidéos sont typiquement sous licence \textit{CC}. \textit{Harvard, Oxford, MIT, Yale, Princeton, HEC Paris, Stanford, Berkeley}... Tous les grands noms y sont.
La distribution se fait soit sur le site web de l'université, soit au moyen de flux \texttt{RSS}, soit dans \textit{iTunes}, il y en a aussi pas mal sur \textit{Youtube}, etc.
Notre Université Catholique de Louvain est un peu en retard, mais les podcasts devraient arriver.
Notons également que le professeur O. \textsc{Bonaventure} a reçu récemment un prix américain pour la publication de syllabus sous licence libre. %% mentionner le syllabus libre de O. Bonaventure?

\paragraph{\textit{Internet archive}}
Sur le site \url{archive.org} vous trouvez une collection de presque toutes ces \oe uvres libres mentionnées ci-dessus et surtout plus de 500.000 films, un million d’enregistrements audio et trois millions de livres.
Si vous voulez, par exemple, revoir comment la presse a réagi lors des événements du 11 septembre, vous le trouvez ici.
Pour alléger le trafic sur leurs serveurs, l'archive fait usage du protocole \textit{Torrent}, souvent maladroitement lié aux échanges illégaux de fichiers.

\textit{Internet archive} s'est aussi donné le but d'archiver le web. Le \textit{Wayback machine} permet ainsi de consulter des sites web tels qu'ils étaient à des moments précis dans le passé.

\paragraph{Sintel}
Sintel est un court métrage libre de 15 minutes réalisé en 2010 par \textsc{Tom Rosendaal} et toute une équipe d'artistes.
L'histoire tourne autour d'un dragon orphelin \textit{Scales} et de \textit{Sintel} qui s'en occupe jusqu'à ce qu'il se fasse enlever par un dragon adulte.
%Nous ne racontons pas la fin.
Le logiciel libre \textit{Blender} est à la source des images animées.
Vous pouvez télécharger le film gratuitement (et légalement) ainsi que d'autres courts métrages qui sont produits chaque année.

\end{document}
