\documentclass[10pt]{../fiche}

\begin{document}

\titlellnux{8}{L'obsolescence programmée}
\section*{L'obsolescence programmée}

De nos jours, les appareils informatiques ont une durée de vie de plus en plus courte. Ainsi, les ordinateurs qui duraient auparavant une dizaine d'année, ne tiennent plus que 5 à 8 ans (1). On appelle ce phénomène l'obsolescence programmée. Commençons tout d'abord par définir  cette obsolescence programmée selon les normes de la commission
Européenne (2) comme :
``\textit{un stratagème par lequel un bien verrait sa durée
normative sciemment réduite dès sa conception, limitant
ainsi sa durée d'usage pour des raisons de modèle économique.
Ces techniques peuvent notamment inclure
... une impossibilité de réparer ou une non-compatibilité
logicielle}''

On peut ouvertement se demander si cette obsolescence est néfaste et de quelles manières. Tout d'abord, cette amoindrissement programmée de la durée de vie crée une demande qui entraine l'excavation de grandes quantités de terre engendrant le défrichage des sols, l'élimination de la végétation et la destruction des terres fertiles. De plus, nos matériaux électroniques nécessitent une grande quantité de métaux (or, argent, étain, silicium, ...) qui sont des ressources non renouvelables dont le recyclage reste minoritaire. Ce modèle économique n'est donc pas viable à long terme sous peine d'indisponibilité de ressources.

En ce qui concerne le libre (donc le domaine de l'informatique), deux types d'obsolescence peuvent être associées (3) :
\begin{itemize}
  \item L'indirecte : ``\textit{Certains produits deviennent obsolètes alors qu'ils sont totalement fonctionnels de par le fait que les produits consommables associés ne sont pas ou plus disponibles sur le marché}''

  \item Par incompatibilité : ``\textit{..Cette technique vise à rendre un produit inutile par le fait qu'il n'est plus compatible
    avec les versions ultérieures. Dans le cas d'un logiciel, le changement de format de fichier entre deux versions successives d'un même programme suffira à rendre les anciennes versions obsolètes puisque non compatibles avec le nouveau standard}''
\end{itemize}

Les logiciels libres peuvent apporter une solution à ces problématiques. En effet, un logiciel libre suit quatre directives normatives : liberté d'exécuter, liberté d'étudier le fonctionnement du logiciel et de l'adapter à ses besoins, liberté de redistribuer des copies du logiciel, liberté d'améliorer le logiciel et de publier ces améliorations (4). Ces garanties d'utilisation permettent ainsi un accès au code source et son adaptation continue à de nouveau système d'exploitation. De cette façon, une incompatibilité due au format de fichiers peut être modernisé à ses nouveaux standards. Le cas de l'obsolescence indirecte concerne entre autre les systèmes d'exploitation plus anciens. L'OS est toujours bien fonctionnel, tout comme l'ordinateur, mais le fabricant ne fournit plus de support technique. Un exemple récent est celui de \textsf{Windows XP} : en Avril 2014, \textsf{Microsoft} met fin au support d'\textsf{XP}. \textit{A contrario}, la plupart des OS libres (\textsf{Ubuntu, Fedora, Mint,...}) sortent régulièrement des mises à jour avec un "Long-Term Support (LTS)". Cela signifie qu'après chaque nouvelle édition, le système est supporté de manière fiable pour une période donnée (5 ans pour \textsf{Ubuntu} (5)). Il est important de préciser que l'OS ne change pas, ne causant pas de problèmes de compatibilité. Dans le cas de \textsf{Windows XP}, l'utilisateur n'a pas l'opportunité d'actualiser son système d'exploitation librement; il est contraint par le fournisseur de changer de version, i.e. \textsf{Windows 7}.

Toutefois, le monde du libre n'échappe pas complètement à la thématique de l'obsolescence. En effet, des performances trop élevées nécessitent des composants adaptés au sein des ordinateurs pour supporter la consommation physique de logiciels. Ainsi, ces besoins rendent le matériel inutilisable en pratique aussi dans le cas des logiciel libre (même si ceux-ci sont supportés). \footnote{voir l'exemple de \textsf{KDE} sous \textsf{Linux} : \url{http://forum.ubuntu-fr.org/viewtopic.php?id=380878}}. Le libre représente néanmoins une alternative incontestable au modèle de consommation actuel. Une correspondance entre des politiques de recyclage et de production adéquates et une utilisation à long termes des logiciel libres permet une réelle vision à long terme de l'informatique. Ainsi, on peut imaginer un cycle de vie durable allant de la production à l'utilisation et enfin à l'élimination, en passant par la mise en vente et le transport.

\section*{Source :}
\begin{enumerate}
\item \url{consoglobe.com/obsolescence-programmee-appareils-cg}

\item Centre Européen de la Consommation;
\textsf{L'obsolescence programmée ou les dérives de la société de consommation}

\item site internet Wikipédia l’encyclopédie en ligne
\url{http://fr.wikipedia.org/wiki/Obsolescence_programm\%C3\%A9e#Obsolescence_par_incompatibilit.C3.A9}

\item \url{https://aful.org/ressources/licences-libres}

\item \url{https://wiki.ubuntu.com/LTS}
\end{enumerate}
\end{document}
