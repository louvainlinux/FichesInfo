\documentclass[12pt]{../fiche}

\begin{document}

\titlellnux{11}{Le Libre, kesskecé ?}

Salut à toi futur ami du Libre ! 
\vspace{0.8em}

Hé oui, le Louvain-Li-Nux a encore pensé à vous. Les amis, il est temps de sortir de l'ignorance et de vous plonger dans la connaissance de ce qu'est le Libre !
\vspace{0.8em}

Le Libre, en gros, c'est une license pour publier des propriétés intellectuelles --- souvent des logiciels, mais pas uniquement --- et de manière plus générale, un état d'esprit. On parle généralement des \og{}quatre grands piliers\fg{} qui décrivent ce qu'est le Libre :
\begin{itemize}
\item la liberté d'utilisation ;
\item l'utilisateur a accès à la façon dont le programme ou l'objet a été fait, il peut consulter la “recette de fabrication” pour l'étudier ;
\item l'utilisateur peut modifier cette recette. Dans le cas d'un logiciel par exemple, il peut modifier le code source pour l'améliorer ou l'adapter à ses besoins ;
\item l'utilisateur peut redistribuer le produit de sa modification.
\end{itemize}

Une licence qui ne propose pas ces quatre piliers est appelée licence propriétaire. Par exemple, Windows, Microsoft Office, Photoshop sont des logiciels propriétaires.

\vspace{0.8em}

La plupart des gens utilisent des logiciels libres, parfois sans même le savoir. Nous pouvons en citer quelques-uns : Firefox, VLC, Gimp, LibreOffice, jDownloader. Ce sont des logiciels qui ne sont pas installés lorsqu'on achète un ordinateur mais qu'on peut facilement trouver sur internet. Nous pouvons donc dire que les logiciels libres sont très présents et sont fiables.
\vspace{0.8em}

Le Libre est parfois également mal compris. Quelques idées du sens commun peuvent ainsi être écartées :
\begin{itemize}
\item un logiciel libre n'est pas libre de droit. Il est soumis à une license qui peut imposer certaines restrictions comme citer les auteurs originaux par exemple. C'est comme une forme de contrat qui impose des contraintes mais en garantissant les quatre piliers du Libre ;
\item un logiciel libre n'est pas toujours gratuit ! Des sociétés peuvent très bien vendre des logiciels libres, souvent en y ajoutant des services ;
\item un logiciel libre peut donc être commercialisé.
\end{itemize}

\vspace{0.8em}

Étant donné qu'un logiciel propriétaire ne divulgue pas la manière dont il fonctionne, il est possible --- et c'est souvent le cas --- que la société qui le publie s'en serve également pour récupérer, souvent à votre insu, des informations à votre sujet, ainsi que pour restreindre et censurer votre usage du logiciel. Dans certains cas, c'est même leur principale source de revenus, comme dans le cas de Facebook.

Si ce problème vous semble peu important, réfléchissez au fait que les logiciels de vote électronique sont presque toujours propriétaires et fermés. Cela implique qu'une société privée possède le contrôle sur le fonctionnement du programme, et qu'aucune garantie n'est offerte en pratique sur le fait que ledit programme compte les voix correctement.

Il y a beaucoup d'autres problèmes liés à l'utilisation de logiciels propriétaires, trop pour être listés dans cette courte fiche info. On vous invite à vous renseigner sur le lien donné en bas de page.

\vspace{0.8em}

Si vous avez des questions sur le Libre, n'hésitez pas à passer au kot lors de nos permanences informatiques le lundi de 18h30 à 22h. Nous serons ravis de discuter autour du Libre, de vous aider à installer Linux ou votre programme libre préféré ! Tenez-vous également au courant de nos nombreuses activités qui vont seront plus qu'utiles pour apprendre à utiliser votre ordinateur d'une façon différente et plus réfléchie !
\vspace{0.8em}

Pour plus d'infos : \url{www.gnu.org/philosophy/philosophy.fr.html}



\end{document}
